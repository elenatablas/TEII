
\subsection{Pasos VirtualBox}

%%%%%%%%%%%%%%%%%%%%%%%%%%%%%%%%%%%%%%%%%%%%%%%%%%%%%%%%%%%%%%%%%%%%%%%%%%%%%%%%%%%%%%%%%%%%%%%%%%%%%%%%%%%%%%%%%%%%%%%%%%%%%%%%%%%%%%%%%%%%%%
\subsubsection{Crear una VM mediante el Shell de VirtualBox, Vboxmanage: 1GB de ram, 2 cpus, 8 GB de disco duro SATA.}

\par \textbf{El primer paso} para crear una máquina virtual es usar el comando \texttt{createvm},
que su única función es construir un archivo \texttt{XML} que define la nueva maquina virtual.
La opción \texttt{--register} sirve para importar la definición de la maquina virtual
dentro de Oracle VM VirtualBox, aunque también se puede realizar con el comando 
\texttt{registervm}. El uso del parámetro \texttt{--name} es obligatorio y en este caso 
nombraré mi primera VM como \textit{teii-server}, adicionalmente incluyo el tipo de 
sistema operativo que tengo intención de instalar en esta máquina, como se muestra en 
la Figura \ref{fig:vboxmanage-createvm}.
%%% IMAGEN VBOXMANAGE CREATEVM %%%
\begin{figure}[H]
    \includegraphics[width=\textwidth]{vboxmanage-createvm}
    \centering
    \caption{Definición e importación de la VM \textit{teii-server} dentro de \texttt{VirtualBox}.}
    \label{fig:vboxmanage-createvm}
 \end{figure}
\par Esta máquina virtual está vacía, en los siguientes pasos tengo que 
configurar la CPU, RAM, red y el disco duro. Además, en la siguiente sección 
instalo el sistema operativo. 
%%% IMAGEN VBOXMANAGE SHOWVMINFO INICIAL%%%
\begin{figure}[H]
    \includegraphics[width=\textwidth]{vboxmanage-showvminfo-inicial}
    \centering
    \caption{Información inicial de la \texttt{VM} \textit{teii-server}.}
    \label{fig:vboxmanage-showvminfo-inicial}
 \end{figure}
\par El comando \texttt{showvminfo} muestra la 
información de la VM, en este caso es la configuración inicial por defecto y 
se puede comprobar en la Figura \ref{fig:vboxmanage-showvminfo-inicial} como la configuración impuesta no es la que quiero.
Por tanto, en \textbf{el segundo paso} modifico el número de CPUs y el tamaño de la RAM con el comando 
\texttt{modifyvm} y compruebo que la información se ha configurado correctamente en la Figura \ref{fig:vboxmanage-showvminfo-segundo-paso}.
\begin{listing}[style=consola]
    $ VBoxManage modifyvm teii-server --cpus 2 --memory 1024
\end{listing}
%%% IMAGEN VBOXMANAGE SHOWVMINFO SEGUNDO PASO %%%
\begin{figure}[H]
    \includegraphics[width=\textwidth]{vboxmanage-showvminfo-segundo-paso}
    \centering
    \caption{Información modificada de la VM \textit{teii-server}.}
    \label{fig:vboxmanage-showvminfo-segundo-paso}
 \end{figure}
\par Además, configuro el adaptador de red con el mismo comando que antes para que 
la máquina virtual pueda interactuar con el host y otras máquinas virtuales conectadas
a la misma red. Este paso es útil para que otras máquinas puedan acceder a este servidor web con servicio \texttt{nginx}.
\begin{listing}[style=consola]
    $ VBoxManage modifyvm teii-server --nic1 hostonly --hostonlyadapter1 vboxnet2
\end{listing}

\par \textbf{El tercer paso} es la creación de un disco duro virtual. Como en un 
ordenador real, la VM necesita uno para poder bootear y almacenar información del sistema y del usuario. 
Para ello debo utilizar tres comandos.
\begin{itemize}
    \item Comando \texttt{createhd}: crea una imagen de disco en formato \texttt{VDI}.
    Si el directorio que se le pasa como parámetro no existe, \texttt{VirtualBox} lo crea.
    %%% IMAGEN VBOXMANAGE CREATEHD %%%
    \begin{figure}[H]
        \includegraphics[width=\textwidth]{vboxmanage-createhd}
        \centering
        \caption{Creación de una imagen de disco de 8GB.}
        \label{fig:vboxmanage-createhd}
    \end{figure}
    \item Comando \texttt{storagectl}: añade un controladore de alamacenamiento, en este caso elijo \texttt{Serial ATA (SATA)}.
\begin{listing}[style=consola]
    $ VBoxManage storagectl teii-server  --name "SATA Controller" --add sata --bootable on
\end{listing}
    \item Comando \texttt{storageattach}: asocia el nuevo disco duro virtual con el controlador.
\begin{listing}[style=consola]
    $ VBoxManage storageattach teii-server --storagectl "SATA Controller" --port 0 --device 0 --type hdd --medium /Users/elenatablas/VirtualBox\ VMs/teii-server/teii-server.vdi
\end{listing}
\end{itemize}
\par Compruebo que la información del nuevo disco duro virtual es correcta como indica la Figura \ref{fig:vboxmanage-showmediuminfo-inicial}.
%%% IMAGEN VBOXMANAGE SHOWMEDIUMINFO INICIAL %%%
\begin{figure}[H]
    \includegraphics[width=\textwidth]{vboxmanage-showmediuminfo-inicial}
    \centering
    \caption{Información del nuevo disco duro virtual usado en la VM \textit{teii-server}.}
    \label{fig:vboxmanage-showmediuminfo-inicial}
 \end{figure}

%%%%%%%%%%%%%%%%%%%%%%%%%%%%%%%%%%%%%%%%%%%%%%%%%%%%%%%%%%%%%%%%%%%%%%%%%%%%%%%%%%%%%%%%%%%%%%%%%%%%%%%%%%%%%%%%%%%%%%%%%%%%%%%%%%%%%%%%%%%%%%
\subsubsection{Instalar ubuntu-server en la máquina. Instalarle el servicio nginx a la máquina.}
\par \textbf{El primer paso} es descargar la imagen iso en el ordenador, en este caso he descargado
la versión \texttt{ubuntu-16.04.7-server-amd64}.
\par \textbf{El siguiente paso} es configurar el arranque de la maquina virtual para que 
la instalación empiece cuando la VM arranque por primera vez. 
\par \textbf{El tercer paso} es crear un dispositivo CD/DVD virtual y conectarlo. Además, 
necesita un controlador. Utilizaré los mismos comandos descritos cuando la creación del disco duro virtual.

\begin{listing}[style=consola]
    $ VBoxManage storagectl teii-server --name "IDE Controller" --add ide
    $ VBoxManage storageattach teii-server --storagectl "IDE Controller" --port 0  --device 0 --type dvddrive --medium /Users/elenatablas/Downloads/ubuntu-16.04.7-server-amd64.iso
\end{listing}
\par Compruebo que la información del controlador IDE como indica la Figura \ref{fig:vboxmanage-showvminfo-controller}.
%%% IMAGEN VBOXMANAGE SHOWVMINFO IDE CONTROLLER %%%
\begin{figure}[H]
    \includegraphics[width=\textwidth]{vboxmanage-showvminfo-controller}
    \centering
    \caption{Información del controlador \texttt{IDE} usado en la VM \textit{teii-server}.}
    \label{fig:vboxmanage-showvminfo-controller}
 \end{figure}
 \par \textbf{El cuarto paso} es arrancar la maquina virtual usando el comando \texttt{startvm}, Figura \ref{fig:vboxmanage-startvm} 
 y configurar los pasos iniciales.
 %%% IMAGEN VBOXMANAGE STARTVM %%%
\begin{figure}[H]
    \includegraphics[width=\textwidth]{vboxmanage-startvm}
    \centering
    \caption{Arrancar por primera vez la VM \textit{teii-server}.}
    \label{fig:vboxmanage-startvm}
\end{figure}
\par \textbf{El quinto paso} es apagar la maquina virtual con cualquier comando de estos dos:
\begin{listing}[style=consola]
    $ VBoxManage controlvm teii-server acpipowerbutton
    $ VBoxManage controlvm teii-server poweroff
\end{listing}
\par \textbf{El sexto paso} es eliminar el DVD de la configuración de la VM porque el sistema operativo ya está instalado.
\begin{listing}[style=consola]
    $ VBoxManage storageattach teii-server --storagectl "IDE Controller" --port 0 --device 0 --type dvddrive --medium none
\end{listing}
 %%% IMAGEN VBOXMANAGE STARTVM %%%
 \begin{figure}[H]
    \includegraphics[width=\textwidth]{vboxmanage-eliminar-dvd}
    \centering
    \caption{Eliminación del DVD de la configuración de la VM \textit{teii-server}.}
    \label{fig:vboxmanage-eliminar-dvd}
\end{figure}
\par \textbf{Por último}, arranco la máquina virtual para ver que funciona correctamente e instalo el servicio nginx.
\begin{listing}[style=consola]
    $ VBoxManage controlvm teii-server acpipowerbutton
    $ VBoxManage controlvm teii-server poweroff
\end{listing}
 %%% IMAGEN INSTALACIÓN NGINX %%%
 \begin{figure}[H]
    \includegraphics[width=\textwidth]{instalar-nginx}
    \centering
    \caption{Instalación del servicio \texttt{nginx} en la VM \textit{teii-server}.}
    \label{fig:instalar-nginx}
\end{figure}
 
%%%%%%%%%%%%%%%%%%%%%%%%%%%%%%%%%%%%%%%%%%%%%%%%%%%%%%%%%%%%%%%%%%%%%%%%%%%%%%%%%%%%%%%%%%%%%%%%%%%%%%%%%%%%%%%%%%%%%%%%%%%%%%%%%%%%%%%%%%%%%%
\subsubsection{Instalar ubuntu-desktop en otra máquina. Instalar un navegador web.}
\par Los comandos de esta sección son los mismos mencionados en el proceso de creación e instalación de la 
máquina virtual teii-server. Por tanto, las explicaciones se dan por dadas y en está sección 
muestro todos los comandos necesarios y la instalación de un navegador web.
\begin{listing}[style=consola]
    $ VBoxManage createvm --name Linux-Ubuntu --ostype Ubuntu_64 --register
    $ VBoxManage modifyvm Linux-Ubuntu --memory 2048
    $ VBoxManage createhd --filename /home/alumno/disco.vdi --size 1024
    $ VBoxManage storagectl Linux-Ubuntu --name "SATA Controller" --add sata -- controller IntelAhci
    $ VBoxManage storageattach Linux-Ubuntu --storagectl "SATA Controller" -- type HDD --port 1 --device 0 --medium /home/alumno/disco.vdi
    $ VBoxManage storagectl Linux-Ubuntu --name "IDE Controller" --add ide -- controller PIIX4
    $ VBoxManage storageattach Linux-Ubuntu --storagectl "IDE Controller" -- type DVDDrive --port 1 --device 0 --medium /home/alumno/ubuntu-18.04.4-server-amd64.iso
\end{listing}


%%%%%%%%%%%%%%%%%%%%%%%%%%%%%%%%%%%%%%%%%%%%%%%%%%%%%%%%%%%%%%%%%%%%%%%%%%%%%%%%%%%%%%%%%%%%%%%%%%%%%%%%%%%%%%%%%%%%%%%%%%%%%%%%%%%%%%%%%%%%%%
\subsubsection{Comprobar que podemos acceder al servidor web desde un navegador en el host.}
\par En la Figura \ref{fig:comprobacion-host}, a la izquierda se puede ver la máquina virtual 
corriendo con el servicio \texttt{nginx} activado y que su dirección IP es la \texttt{192.168.58.3}.
Para acceder al servidor web desde mi host (macOS), abro el navegador web Opera y escribo la url \url{http://192.168.58.3}
como se aprecia en el lado derecho de la Figura \ref{fig:comprobacion-host}.

%%% IMAGEN COMPROBACIÓN HOST %%%
 \begin{figure}[H]
    \includegraphics[width=\textwidth]{comprobacion-host}
    \centering
    \caption{Comprobación acceso al servidor web \textit{teii-server} desde el host \texttt{macOS}.}
    \label{fig:comprobacion-host}
\end{figure}


%%%%%%%%%%%%%%%%%%%%%%%%%%%%%%%%%%%%%%%%%%%%%%%%%%%%%%%%%%%%%%%%%%%%%%%%%%%%%%%%%%%%%%%%%%%%%%%%%%%%%%%%%%%%%%%%%%%%%%%%%%%%%%%%%%%%%%%%%%%%%%
\subsubsection{Comprobar que podemos acceder al servidor web desde un navegador en la máquina guest desktop.}

\begin{listing}[style=consola]
    $ 
\end{listing}

%%%%%%%%%%%%%%%%%%%%%%%%%%%%%%%%%%%%%%%%%%%%%%%%%%%%%%%%%%%%%%%%%%%%%%%%%%%%%%%%%%%%%%%%%%%%%%%%%%%%%%%%%%%%%%%%%%%%%%%%%%%%%%%%%%%%%%%%%%%%%%
\subsubsection{Transformar el disco de la máquina de formato .vdi a formato qcow2.}

\begin{listing}[style=consola]
    $ 
\end{listing}


%%%%%%%%%%%%%%%%%%%%%%%%%%%%%%%%%%%%%%%%%%%%%%%%%%%%%%%%%%%%%%%%%%%%%%%%%%%%%%%%%%%%%%%%%%%%%%%%%%%%%%%%%%%%%%%%%%%%%%%%%%%%%%%%%%%%%%%%%%%%%%
\subsubsection{Redimensionar la maquina original servidor de VirtualBox con la utilidad Vboxmanage: 2GB de RAM y 10GB disco duro.} 

\begin{listing}[style=consola]
    $ 
\end{listing}


%%%%%%%%%%%%%%%%%%%%%%%%%%%%%%%%%%%%%%%%%%%%%%%%%%%%%%%%%%%%%%%%%%%%%%%%%%%%%%%%%%%%%%%%%%%%%%%%%%%%%%%%%%%%%%%%%%%%%%%%%%%%%%%%%%%%%%%%%%%%%%
\subsubsection{Destruir las máquinas virtuales (utilizando comandos).}

\begin{listing}[style=consola]
    $ 
\end{listing}

