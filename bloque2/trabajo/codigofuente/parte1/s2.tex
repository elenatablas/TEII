
\subsection{Pasos libvirt/kvm}

\subsubsection{Crear una VM mediante el Shell de VirtualBox, Vboxmanage: 1GB de ram, 2 cpus, 8 GB de disco duro SATA.}

\begin{listing}[style=consola]
    $ 
\end{listing}

\par El primer paso es crear  una máquina virtual 

\begin{listing}[style=consola]
    $ 
\end{listing}

\subsubsection{Instalar ubuntu-server en la máquina. Instalarle el servicio nginx a la máquina.}

\begin{listing}[style=consola]
    $ 
\end{listing}

\subsubsection{Instalar Ubuntu-desktop en otra máquina. Instalar un navegador web.}

\begin{listing}[style=consola]
    $ 
\end{listing}

\subsubsection{Comprobar que podemos acceder al servidor web desde un navegador en el host.}
\begin{listing}[style=consola]
    $ 
\end{listing}

\subsubsection{Comprobar que podemos acceder al servidor web desde un navegador en la máquina guest desktop.}

\begin{listing}[style=consola]
    $ 
\end{listing}

\subsubsection{Transformar el disco de la máquina de formato .vdi a formato qcow2.}

\begin{listing}[style=consola]
    $ 
\end{listing}

\subsubsection{Redimensionar la maquina original servidor de VirtualBox con la utilidad Vboxmanage: 2GB de RAM y 10GB disco duro.} 

\begin{listing}[style=consola]
    $ 
\end{listing}

\subsubsection{Destruir las máquinas virtuales (utilizando comandos).}

\begin{listing}[style=consola]
    $ 
\end{listing}