%%%%%%%%%%%% INTRODUCCIÓN  %%%%%%%%%%%%

\begin{center}
	{\fboxrule=4pt \fbox{\fboxrule=1pt
		\fbox{\LARGE{\bfseries Introducción}}}} \\
	\addcontentsline{toc}{section}{Introducción}
	\rule{15cm}{0pt} \\
\end{center}

\pagenumbering{roman}
 
\lettrine[lines=3, depth = 0]{E}{n} este documento

\par Para generar mi documentación he utilizado \LaTeX, en especial, he añadido
 en el preambulo el paquete ``\texttt{Sage\TeX}'' que permite incrustar los
 resultados del cálculo \texttt{Sage} en este documento.
\par Para hacer uso de este paquete:
\begin{enumerate}
	\item Instalar \LaTeX en el ordenador donde voy a realizar las prácticas
 	\item Instalar \texttt{Python} y \texttt{Sage} el ordenador donde voy a realizar las prácticas
  	\item Seguir las instrucciones de la documentación \url{https://doc.sagemath.org/html/en/tutorial/sagetex.html}
\end{enumerate}
\par Para generar este documento:
\begin{enumerate}
	\item Ejecutar \LaTeX en mi fichero \texttt{.tex}
 	\item Ejecutar Sage en el fichero generado \texttt{.sage}
  	\item Ejecutar \LaTeX en mi fichero \texttt{.tex} otra vez.
\end{enumerate}


\newpage
\pagenumbering{arabic}