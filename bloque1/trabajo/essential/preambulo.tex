%%%%%% %%%%%% PREAMBULO TEII_B1 %%%%%% %%%%%%

%Magenes
\usepackage[margin=1 in, includefoot]{geometry}

% Acentos, letras, etc
\usepackage[T1]{fontenc}
\usepackage[utf8]{inputenc}
\usepackage[spanish]{babel}
\addto\captionsspanish{
\renewcommand{\chaptername}{}
}
\selectlanguage{spanish}
\usepackage{lettrine, Zallman}
\renewcommand\LettrineFontHook{\Zallmanfamily}

% Caja de texto
\usepackage{parskip}

\usepackage{times}
\usepackage{ulem} % texto tachado
\usepackage{lipsum}
\usepackage[usenames]{color} % color en texto
%Creación de mis propios colores
\definecolor{rojoOscuro}{RGB}{153,0,0}
\definecolor{verdeOscuro}{RGB}{0,153,0}
\usepackage{verbatim} % comentarios
% Letra para código
\usepackage{listings}
\lstdefinestyle{consola}
{basicstyle=\scriptsize\bf\ttfamily,
backgroundcolor=\color{gray75},
}

% Enumerados
\renewcommand{\labelenumii}{\alph{enumii}$)$ }

\usepackage{comment}

% Header and Footer Stuff
\usepackage{fancyhdr}

%hipervínculos, referencias, citas, etc
\usepackage[colorlinks=true, 
		     linkcolor=magenta, 
		     citecolor = black,
		     urlcolor = blue]{hyperref}
\pagestyle{fancy}
\fancyhead[L,RO]{}
\fancyhead[LO,R]{}
%fancy
\renewcommand{\footrulewidth}{1pt}

% Símbolos matemáticos
\usepackage{amsmath}
\usepackage{array}

% Figuras, gráficas, tablas y matrices
\usepackage{float}
\usepackage{subfigure} % varias figuras
\usepackage{graphicx} 
\usepackage{longtable}
\graphicspath{ {images/} }


% Color en las tablas
\usepackage{color}
\usepackage{epsfig}
\usepackage{multirow}
\usepackage{colortbl}
\usepackage[table]{xcolor}
\usepackage{soul}

%https://ondahostil.wordpress.com/2017/05/17/lo-que-he-aprendido-cuadros-de-texto-de-colores-en-latex/
\usepackage{lmodern}
\usepackage{tcolorbox}
\tcbuselibrary{listingsutf8}

% Definir cuadro de ancho del texto
\newtcolorbox{mybox}[1]{colback=purple!5!white,colframe=purple!75!black,fonttitle=\bfseries,title=#1}
% OPCIONES
% colback: color de fondo
% colframe: color de borde
% fonttitle: estilo de título
% title: título de la cuadro o referencia a argumento

% Cuadro estrecho
\newtcbox{cuadro}[1]{colback=blue!5!white,colframe=blue!75!black,fonttitle=\bfseries,title=#1}

% Cuadro numerado para ejemplos
\newtcolorbox[auto counter,number within=subsection]{example}[2][]
{colback=purple!5!white,colframe=purple!75!black,fonttitle=\bfseries, title=Ejemplo~\thetcbcounter: #2,#1}

\newtcolorbox[auto counter,number within=subsection]{ejercicio}[2][]
{colback=orange!5!white,colframe=orange!75!black,fonttitle=\bfseries, title=#2,#1}

\newtcolorbox[auto counter,number within=subsection]{solucion}[2][]
{colback=yellow!5!white,colframe=yellow!75!black,fonttitle=\bfseries, title=#2,#1}

% Esquemas con llaves https://tex.stackexchange.com/questions/164664/how-to-create-an-array-with-both-vertical-and-horizontal-braces-around-the-eleme
\newcommand\undermat[2]{%
\makebox[0pt][l]{$\smash{\underbrace{\phantom{%
\begin{matrix}#2\end{matrix}}}_{\text{$#1$}}}$}#2}

% Licencia
\usepackage[
    type={CC},
    modifier={by-nc-sa},
    version={3.0},
]{doclicense}

% Letra para código http://www.rafalinux.com/?p=599
\definecolor{gray97}{gray}{.97}
\definecolor{gray75}{gray}{.75}
\definecolor{gray45}{gray}{.45}
\definecolor{pblue}{rgb}{0.13,0.13,1}
\definecolor{pgreen}{rgb}{0,0.5,0}
\definecolor{pred}{rgb}{0.9,0,0}
\definecolor{pgrey}{rgb}{0.46,0.45,0.48}

\usepackage{listings}
\lstset{ 
	frame=Ltb,
	framerule=0pt,
	aboveskip=0.5cm,
	framextopmargin=3pt,
	framexbottommargin=3pt,
	framexleftmargin=0.2cm,
	framesep=0pt,
	rulesep=.4pt,
	backgroundcolor=\color{gray97},
	rulesepcolor=\color{black},
%
	stringstyle=\ttfamily,
	showstringspaces = false,
	basicstyle=\small\ttfamily,
	commentstyle=\color{gray45},
	keywordstyle=\bfseries,
%
	numbers=left,
	numbersep=7pt,
	numberstyle=\tiny,
	numberfirstline = false,
	breaklines=true,
}

% minimizar fragmentado de listados
\lstnewenvironment{listing}[1][]
{\lstset{#1}\pagebreak[0]}{\pagebreak[0]}

\lstdefinestyle{consola}
{basicstyle=\scriptsize\bf\ttfamily,
backgroundcolor=\color{gray75},
}

%Llaves
\usepackage{schemata}
% Foto
\usepackage{wrapfig}
% Tablas
\usepackage{multirow}
\usepackage{tikz}
\def\checkmark{\tikz\fill[scale=0.4](0,.35) -- (.25,0) -- (1,.7) -- (.25,.15) -- cycle;} 

%https://ondahostil.wordpress.com/2017/05/17/lo-que-he-aprendido-cuadros-de-texto-de-colores-en-latex/
\usepackage{lmodern}
\usepackage{tcolorbox}
\tcbuselibrary{listingsutf8}

\setlength{\parindent}{0.5cm}
\lhead[\leftmark]{Pérez González-Tablas, Elena}
\rhead[Nombre Autor]{Tecnologías Específicas de la Ingeniería Informática, 2021/22}
\newtcolorbox[auto counter,number within=subsection]{ejer}[1][]
{colback=blue!5!white,colframe=blue!75!black,fonttitle=\bfseries, title=#1}

\usepackage{sagetex}

